En conclusion, ce projet a été une opportunité exceptionnelle pour appliquer concrètement les compétences théoriques acquises en programmation. Il nous a introduits à des outils de développement avancés tels que Makefile et LaTeX, enrichissant ainsi notre boîte à outils technique.
\\
Au cours de ce projet, nous avons renforcé nos compétences en travail d'équipe, notamment dans la répartition efficace des tâches et l'utilisation collaborative de git et GitLab. Cette expérience nous a permis de maîtriser des langages de programmation turing-complets, comme le C et le LaTeX, élargissant notre perspective de programmation.
\\
Ce projet a été particulièrement enrichissant dans la compréhension du fonctionnement des compresseurs de fichiers, tout en nous familiarisant avec la manipulation de structures de données complexes telles que les arbres et les files. De plus, nous avons acquis des compétences précieuses dans la réalisation de programmes à la fois performants et optimisés. Cela a inclus l'optimisation de l'utilisation de la mémoire par des opérations bit à bit et l'amélioration de l'efficacité temporelle à travers l'emploi de structures de données appropriées.
\\
En somme, ce projet a été une expérience formatrice, nous dotant d'une compréhension plus profonde des principes de la programmation et de l'ingénierie logicielle.
