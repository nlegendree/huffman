\begin{algorithme}


    \begin{enregistrement}{TableDeCodage}
        \champEnregistrement{nbElements}{\naturel}
        \champEnregistrement{lesElements}{Table}
    \end{enregistrement}

    \begin{enregistrement}{Table}
        \champEnregistrement{elements}{\tableau{1..MAX}{de}{Element}}
        \champEnregistrement{codes}{\tableau{1..MAX}{de}{CodeBinaire}}
    \end{enregistrement}


    \fonction{tableDeCodage}
        {}{TableDeCodage}
        {}{resultat : TableDeCodage}
    {
        \affecter{resultat.nbElements}{0}
        \retourner{resultat}
    }


    \fonction{estVide}
        {table : TableDeCodage}{\booleen}
        {}{}
    {\retourner{table.nbElements = 0}}


    \fonction{elementPresent}
        {table : TableDeCodage, element : Element}{\booleen}
        {}{}
    {
        \pourChaque{table.lesElements.elements}{table}{
            \sialors{element = table.lesElements.elements}{
                {\retourner{vrai}}
            }
        }
        \retourner{faux}
    }


    \fonction{codePresent}
        {table : TableDeCodage, code : CodeBinaire}{\booleen}
        {}{}
    {
        \pourChaque{table.lesElements}{table}{
            \sialors{table.lesElements.codes = code}{
            {\retourner{vrai}}
            }
        }
        \retourner{faux}
    }


    \procedure{ajouterElement}{
        \paramEntree{element : Element}
        \paramEntreeSortie{table : TableDeCodage}}
        {ajouterElement(table, elem) et non(elementPresent(table, elem))}
        {}
    {
        \sialors{non(elementPresent(table, element))}{
            \affecter{table.nbElements}{table.nbElements + 1}
            \affecter{table.lesElements.elements}{element}
            }
    }


    \fonction{elementPossedeCode}
        {table : TableDeCodage, element : Element}{\booleen}
        {elementPresent(table, elem)}
        {}
    {
        \pourChaque{table.lesElements}{table}{
            \sialors{obtenirCodeElement(table, element) = table.lesElements.codes[element]}{
            {\retourner{vrai}}
            }
        }
        \retourner{faux}
    }


    \procedure{assignerCodeElement}{
        \paramEntree{element : Element, code : CodeBinaire}
        \paramEntreeSortie{table : TableDeCodage}}
        {non(codePresent(table, code)) et elementPresent(table, elem)}
        {}
    {
        \sialors{non(codePresent(table, code)) et (elementPresent(table, element))}{
            \pourChaque{table.lesElements.elements}{table}{
                \sialors{table.lesElements.elements = element}{
                    {\affecter{table.lesElements.codes[element]}{code}}
                }
            }
        }
    }


    \fonction{obtenirCodeElement}
        {table : TableDeCodage, element : Element}{CodeBinaire}
        {}{}
    {
        \pourChaque{table.lesElements.elements}{table}{
            \sialors{(table.lesElements.elements = element) et (codePresent(table, table.lesElements.codes[element]))}{
                {\retourner{table.lesElements.codes[element]}}
            }
        }
    }


    \fonction{obtenirElementCode}
        {table : TableDeCodage, code : CodeBinaire}{Element}
        {elementPossedeCode(table, elem) et elementPresent(table, elem)}
        {}
    {
        \pourChaque{table.element}{table}{
            \sialors{(table.lesElements.elements[code] = element) et (codePresent(table, code))}{
            {\retourner{table.lesElements.elements[code]}}
            }
        }
    }
    

\end{algorithme}