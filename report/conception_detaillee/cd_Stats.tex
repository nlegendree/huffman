\begin{algorithme}


    \begin{enregistrement}{UneStatistique}
        \champEnregistrement{nbOccurence}{\naturel}
        \champEnregistrement{element}{Element}
    \end{enregistrement}

    \begin{enregistrement}{Statistiques}
        \champEnregistrement{nbOccurenceTotale}{\naturel}
        \champEnregistrement{lesStatistiques}{\tableau{1..MAX}{de}{UneStatistique}}
    \end{enregistrement}


    \fonction{statistiques}
        {ensemble : ensemble$<$Element$>$}{Statistiques}
        {}
        {resultat : Statistiques
        elem : Element
        i : \naturel}
    {
        \affecter{i}{1}
        \affecter{resultat.nbOccurenceTotale}{0}
        \pourChaque{elem}{ensemble}{
            affecter{resultat.lesStatistiques[i].nbOccurence}{0}
            affecter{resultat.lesStatistiques[i].lesStatistiques}{elem}
            affecter{i}{i+1}
        }
        \retourner{resultat}
    }

a corriger à partir de là

    \fonction{obtenirOccurenceElement}
        {statistiques : Statistiques, element : Element}{\naturel}
        {}{occurences : \naturel}
    {
        \pourChaque{statistiques.lesElements}{statistiques}{
            \sialors{statistiques.lesElements = element}{
                {\affecter{occurences}{occurences + 1}}
            }
        }
        \retourner{occurences}
    }



    \procedure{incrementerOccurenceElement}{
        \paramEntree{element : Element}
        \paramEntreeSortie{statistiques : Statistiques}}
    {}{}
    {
        \pourChaque{statistiques.lesElements}{statistiques}{
            \sialors{statistiques.lesElements = element}{
                {\affecter{occurences}{occurences + 1}}
            }
        }
    }


\end{algorithme}
