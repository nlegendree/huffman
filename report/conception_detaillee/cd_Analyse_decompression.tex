\begin{algorithme}
    \fonction{decompresser}{nom:\chaine}{}{}{stats:Statistiques$<$Octet$>$,arbre:ArbreDeHuffman$<$Octet$>$}%
    {
        sialors{estUnFichierCompresse(nom)}
        {
            \affecter{stats}{obtenirStatistiques(nom)}
            \affecter{arbre}{creerArbre(stats)}
            \instruction{decompresserFichier(source,table,stats)}
        }
    }

    \fonction{estUnFichierCompresse}{nom:\chaine}{\booleen}{}{}
    {
        %vérifie le code d'authentification
    }

    \fonction{obtenirStatistiques}{nom:\chaine}{Statistiques$<$Octet$>$}{}{fichier:FichierBinaire,stats:Statistiques$<$Octet$>$}%
    {
        \affecter{fichier}{fichierBinaire(nom)}  
        \instruction{ouvrir(fichier,lecture)}
        \affecter{stats}{statistiques()}
        \tantque{}%à décider selon comment sont les stats dans la destination
        %à décider si on code les 256 dans l'ordre (opti bcp de carac différent), ou uniquement un octet et son incrémentation à chaque fois (opti peu de carac diff)
        %pas besoin de tant que si 256 codés car taille fixe, plus facile à implémenter
        {
            %repérer l'octet, ou octet suivant selon comment c'est codé
            \instruction{incrementerOccurenceElement(stats,lireOctet(fichier))}
        }
        \instruction{fermer(fichier)}
        \retourner{stats}
    }

    \fonction{creerArbre}{stats:Statistiques$<$Octet$>$}{ArbreDeHuffman$<$Octet$>$}{}{i:\naturel , file:FileDePriorité$<$ArbreDeHuffmann$>$}%a adapter
    {
        \affecter{file}{fileDePrioriteVide()}
        \pour{i}{0}{255}{}%
        {
            \sialorssinon{obtenirOccurenceElement(octet(i))$\ne$0}
            {
                \instruction{ajouterElement(file,arbreDeHuffmann(octet(i),obtenirOccurenceElement(octet(i))))}
            }{}
        }
        \tantque{longueur(file)$>$1}
        {
            \affecter{arbreGauche}{obtenirDernier(file)}
            \instruction{supprimerDernier(file)}
            \affecter{arbreDroit}{obtenirDernier(file)}
            \instruction{supprimerDernier(file)}
            \instruction{ajouterElement(file,combiner(arbreGauche,arbreDroit))}
        }
        \retourner{obtenirDernier(file)}
    }  

    \procedure{descendreArbre}{\paramEntree{arbre:ArbreDeHuffman$<$Octet$>$,code:CodeBinaire},\paramEntreeSortie{table:TableDeCodage$<$Octet$>$}}{}{codeTemp:codeBinaire}%à adapter
    {
        \sialorssinon{estUneFeuille(arbre)}%
        {
            \instruction{ajouterElement(table,obtenirElement(arbre),code)}
        }%
        {
            \affecter{codeTemp}{code}
            \instruction{ajouterBit(code,Bit1)}
            \instruction{ajouterBit(codeTemp,Bit0)}
            \instruction{descendreArbre(obtenirArbreDroit(arbre),code,table)}
            \instruction{descendreArbre(obtenirArbreGauche(arbre),codeTemp,table)}
        }
    }
    \procedure{decompresserFichier}{\paramEntree{nom:\chaine,table:tableDeCodage$<$Octet$>$,stats:Statistiques$<$Octets$>$}}{}{}%
    {
        %à voir selon comment c'est fait, surement juste decendre l'arbre en décrémentant au fur et à mesure, et en fonction de la table de codage, à discuter
    }

\end{algorithme}
