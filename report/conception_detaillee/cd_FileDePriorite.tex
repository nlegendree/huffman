\begin{algorithme}
\begin{enregistrement}{FileDePriorite} %nom du type
    \champEnregistrement{tab}{\tableauUneDimension{1..NOMBRE_MAX_ELEMENT}{de}{Element}} %contenu %en C attention aux indices qui commencent à 0
    \champEnregistrement{tailleActuelle}{\naturel}
\end{enregistrement}
\end{algorithme}

\begin{algorithme}
\fonction{fileDePrioriteVide} %nom de la fonction
    {}{FileDePriorite} %param entree, param sortie
    {} %precond
    {temp : FileDePriorite} %var locales
{   \affecter{temp.tailleActuelle}{0}
    \retourner{temp}} %contenue
\end{algorithme}

\begin{algorithme}
\procedure{ajouterElement}{
    \paramEntree{element : Element, indice : \naturelNonNul}
    \paramEntreeSortie{file : FileDePriorite}}
{} %precond
{i : \naturelNonNul} %var locales (i à placer)
{   \affecter{file.tailleActuelle}{file.tailleActuelle + 1}
    \pour{i}{file.tailleActuelle}{indice+1}{-1}{
        \affecter{file.tab[i]}{file.tab[i-1]}}
    \affecter{file.tab[indice]}{element}} %contenue
\end{algorithme}

\begin{algorithme}
\procedure{supprimerdDernier}{
    \paramEntreeSortie{file : FileDePriorite}}
{non(estVide(file))} %precond
{} %var locales
{   \affecter{file.tailleActuelle}{file.tailleActuelle - 1}} %contenue
\end{algorithme} %on laisse le dernier élément dans le tableau mais on ne s'en occupe plus, il sera écraser ou ignoré si besoin gràce à la taille diminuée

\begin{algorithme}
\fonction{obtenirDernier} %nom de la fonction
    {file : FileDePriorite}{Element} %param entree, param sortie
    {non(estVide(file))} %precond
    {} %var locales
{   \retourner{file.tab[file.tailleActuelle]}} %contenue
\end{algorithme}

\begin{algorithme}
\fonction{estVide} %nom de la fonction
    {file : FileDePriorite}{\booleen} %param entree, param sortie
    {} %precond
    {} %var locales
{   \retourner{file.tailleActuelle = 0}} %contenue
\end{algorithme}

\begin{algorithme}
\fonction{longueur} %nom de la fonction
    {file : FileDePriorite}{\naturel} %param entree, param sortie
    {} %precond
    {} %var locales
{   \retourner{file.tailleActuelle}} %contenue
\end{algorithme}
