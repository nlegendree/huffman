\begin{algorithme}
    \fonction{decompresser}{nom:\chaine}{}{}{stats:Statistiques$<$Octet$>$,arbre:ArbreDeHuffman$<$Octet$>$}%
    {
        sialors{estUnFichierCompresse(nom)}
        {
            \affecter{stats}{obtenirStatistiques(nom)}
            \affecter{arbre}{creerArbre(stats)}
            \instruction{decompresserFichier(source,arbre)}
        }
    }

    \fonction{estUnFichierCompresse}{nom:\chaine}{\booleen}{}{}
    {
        %vérifie le code d'authentification
    }

    \fonction{obtenirStatistiques}{nom:\chaine}{Statistiques$<$Octet$>$}{}{fichier:FichierBinaire,stats:Statistiques$<$Octet$>$}%
    {
        \affecter{fichier}{fichierBinaire(nom)}  
        \instruction{ouvrir(fichier,lecture)}
        \affecter{stats}{statistiques()}
        \tantque{}%à décider selon comment sont les stats dans la destination
        %à décider si on code les 256 dans l'ordre (opti bcp de carac différent), ou uniquement un octet et son incrémentation à chaque fois (opti peu de carac diff)
        %pas besoin de tant que si 256 codés car taille fixe, plus facile à implémenter
        {
            %repérer l'octet, ou octet suivant selon comment c'est codé
            \instruction{fixerOccurenceElement(octet, nbOccurence, stats)}
        }
        \instruction{fermer(fichier)}
        \retourner{stats}
    }

    \fonction{creerArbre}{stats:Statistiques$<$Octet$>$}{ArbreDeHuffman$<$Octet$>$}{}{i:\naturel , file:FileDePriorité$<$ArbreDeHuffmann$>$}
    {
        \affecter{file}{fileDePrioriteVide()}
        \pour{i}{0}{255}{}%
        {
            \sialorssinon{obtenirOccurenceElement(octet(i))$\ne$0}
            {
                \instruction{ajouterElement(file,arbreDeHuffmann(octet(i),obtenirOccurenceElement(octet(i))))}
            }{}
        }
        \tantque{longueur(file)$>$1}
        {
            \affecter{arbreGauche}{obtenirDernier(file)}
            \instruction{supprimerDernier(file)}
            \affecter{arbreDroit}{obtenirDernier(file)}
            \instruction{supprimerDernier(file)}
            \instruction{ajouterElement(file,combiner(arbreGauche,arbreDroit))}
        }
        \retourner{obtenirDernier(file)}
    }  

    \fonction{descendreArbre}{arbre:ArbreDeHuffman$<$Octet$>$,lesBits:\tableau{1..MAX}{de}{Bit},iDebut:\naturel}{Octet,\naturel}{i:\naturel}{noeudActuel:ArbreDeHuffman}
    {
        \affecter{i}{idébut}
        \affecter{noeudActuel}{arbre}
        \tantque{non estUneFeuille(noeudActuel)}
        {
            \sialorssinon{lesBits{i} = bitA0}{
                \affecter{noeudActuel}{obtenirArbreGauche(noeudActuel)}
            }
            {
                \affecter{noeudActuel}{obtenirArbreDroit(noeudActuel)}
            }
            \affecter{i}{i+1}
        }
        \retourner{obtenirElement(noeudActuel),i}
    }

    \fonction{decoderCodeBinaire}{fichier:FichierBinaire,arbre:arbreDeHuffmann$<$Octet$>$}{fichierBinaire}{}{resultat:fichierBinaire,iOctet, iBit, i, k:\naturel,octets:\tableau{1..MAX}{d'}{Octet},octetAEcrire:Octet,lesBits:\tableau{1..MAX}{de}{Bit}}%plus simples à manipuler que les bits ici
    {
        \affecter{resultat}{fichier}
        \affecter{iOctet}{0}
        \affecter{k}{1}
        %instruction pour se placer au début du code binaire à ajouter
        \tantque{non finFichier(fichier)}%
        {
            \affecter{iOctet}{iOctet+1}
            \affecter{Octets[iOctet]}{lireOctet(fichier)}
        }
        \pour{iBit}{1}{iOctet}{}
        {
            \pour{i}{1}{8}{}
            {
                \affecter{lesBits[i]}{obtenirIemeBit(octets[iBit],i)}
                \affecter{k}{k+1}
            }
        }
        \affecter{i}{1}
        \tantque{i<=k}
        {
            \affecter{octetAEcrire, i}{descendreArbre(arbre,lesBits,i)}
            \instruction{ecrireOctet(fichier,octetAEcrire)}
        }
        \retourner{fichier}
    }

    \fonction{decompresserFichier}{\paramEntree{nom:\chaine,arbre:arbreDeHuffmann$<$Octet$>$,stats:Statistiques$<$Octets$>$}}{}{}%
    {
        \affecter{fichierSource}{fichierBinaire(nom)}
        \instruction{tronquer(nom)}%virer le .zip
        \affecter{fichierDestination}{fichierBinaire(nom)}
        \affecter{fichierDestination}{decoderCodeBinaire(fichierSource,arbre)}
        \instruction{fermer(fichierSource)}
        \instruction{fermer(fichierDestination)}
    }

\end{algorithme}
