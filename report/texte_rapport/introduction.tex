Le présent rapport documente le développement d'un compresseur de Huffman réalisé dans le cadre du cours d'Algorithmique avancée et Programmation C durant la seconde moitié du semestre. Ce projet a été conduit en suivant les bonnes pratiques de développement enseignées en cours, mettant en œuvre des concepts clés tels que les Types Abstraits de Données (TAD), la Conception Préliminaire et Détaillée, le Développement, ainsi que les Tests Unitaires.

Le compresseur/décompresseur de Huffman, au cœur de ce projet, permet de compresser des données en utilisant des codes binaires variables pour représenter les caractères du fichier source, attribuant des codes courts aux caractères les plus fréquents et des codes plus longs aux caractères moins fréquents. Ce mécanisme offre une compression sans perte, où la totalité de l'information du fichier original peut être récupérée après décompression.

Le fonctionnement du programme développé suit une interface en ligne de commande avec deux modes principaux : le mode compression (./huffman c fichier) et le mode décompression (./huffman d fichier.huff).

Ce rapport présentera en détail les choix de conception, le processus de développement, ainsi que les résultats des tests unitaires, offrant une perspective complète sur la réalisation de ce projet.

Le pdf du rapport peut-être regénéré grâce à la commande make exécutée dans le doosier rapport.

Le compresseur de Huffman peut également être compilé grâce à un makefile présent dans le dossier programme.
Les options du makefile :
\begin{itemize}
	\item \textbf{all} (make par défaut) compile le programme, les tests unitaires et la documentation.
    \item \textbf{release} compile uniquement le programme.
	\item \textbf{tests} compile les tests unitaires et les exécutent.
	\item \textbf{doc} compile la documentation Doxygen.
\end{itemize}