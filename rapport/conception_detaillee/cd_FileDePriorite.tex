\begin{algorithme}
\begin{enregistrement}{FileDePriorite} %nom du type
    \champEnregistrement{arbres}{\tableauUneDimension{1..256}{d'}{{ArbreDeHuffman}}} %contenu %en C attention aux indices qui commencent à 0
    \champEnregistrement{tailleActuelle}{\naturel}
\end{enregistrement}
\end{algorithme}

\begin{algorithme}
\fonction{fileDePriorite} %nom de la fonction
    {}{FileDePriorite} %param entree, param sortie
    {} %precond
    {temp : FileDePriorite} %var locales
{   \affecter{\champ{temp}{tailleActuelle}}{0}
    \retourner{temp}} %contenue
\end{algorithme}

\begin{algorithme}
\procedure{ajouterElement}{
    \paramEntree{arbre : ArbreDeHuffman}
    \paramEntreeSortie{file : FileDePriorite}}
{} %precond
{i : \naturelNonNul, prio : \naturelNonNul} %var locales
{   \affecter{prio}{obtenirPondération(arbre)}
    \affecter{\champ{file}{tailleActuelle}}{\champ{file}{tailleActuelle} + 1}
    \affecter{i}{\champ{file}{tailleActuelle}}
    \tantque{obtenirPondération(\champ{file}{arbre}[i]) $\leq$ prio}{
        \affecter{\champ{file}{arbre}[i]}{\champ{file}{arbre}[i-1]}}
        \affecter{i}{i-1}
    \affecter{\champ{file}{arbre}[i]}{arbre}}
\end{algorithme}

\begin{algorithme}
\procedure{supprimerDernier}{
    \paramEntreeSortie{file : FileDePriorite}}
{non(estVide(file))} %precond
{} %var locales
{   \affecter{\champ{file}{tailleActuelle}}{\champ{file}{tailleActuelle} - 1}} %contenue
\end{algorithme} %on laisse le dernier élément dans le tableau mais on ne s'en occupe plus, il sera écraser ou ignoré si besoin gràce à la taille diminuée, sa priorité passe à 0 pour ne pas géner

\begin{algorithme}
\fonction{obtenirDernier} %nom de la fonction
    {file : FileDePriorite}{ArbreDeHuffman} %param entree, param sortie
    {non(estVide(file))} %precond
    {} %var locales
{   \retourner{\champ{file}{arbre}[\champ{file}{tailleActuelle}]}} %contenue
\end{algorithme}

\begin{algorithme}
\fonction{estVide} %nom de la fonction
    {file : FileDePriorite}{\booleen} %param entree, param sortie
    {} %precond
    {} %var locales
{   \retourner{\champ{file}{tailleActuelle} = 0}} %contenue
\end{algorithme}

\begin{algorithme}
\fonction{longueur} %nom de la fonction
    {file : FileDePriorite}{\naturel} %param entree, param sortie
    {} %precond
    {} %var locales
{   \retourner{\champ{file}{tailleActuelle}}} %contenue
\end{algorithme}
